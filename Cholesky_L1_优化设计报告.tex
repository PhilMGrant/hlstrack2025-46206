\documentclass[12pt,a4paper]{article}
\usepackage[UTF8]{ctex}
\usepackage{amsmath,amssymb}
\usepackage{graphicx}
\usepackage{booktabs}
\usepackage{multirow}
\usepackage{longtable}
\usepackage{array}
\usepackage{listings}
\usepackage{xcolor}
\usepackage{hyperref}
\usepackage{geometry}
\geometry{a4paper,left=2.5cm,right=2.5cm,top=2.5cm,bottom=2.5cm}

\title{FPGA创新设计大赛 AMD赛道命题式赛道 - 设计报告}
\author{}
\date{}

\lstset{
    language=C++,
    basicstyle=\ttfamily\small,
    keywordstyle=\color{blue},
    commentstyle=\color{green},
    stringstyle=\color{red},
    numbers=left,
    numberstyle=\tiny\color{gray},
    stepnumber=1,
    numbersep=5pt,
    backgroundcolor=\color{white},
    showspaces=false,
    showstringspaces=false,
    showtabs=false,
    frame=single,
    rulecolor=\color{black},
    tabsize=4,
    captionpos=b,
    breaklines=true,
    breakatwhitespace=false,
    escapeinside={\%*}{*)}
}

\begin{document}

\maketitle

\section{项目概述}

\subsection{项目背景}

本项目基于AMD PYNQ-Z2平台,针对Cholesky分解算法进行FPGA硬件加速优化。Cholesky分解是线性代数中的重要算法,广泛应用于信号处理、机器学习、金融建模等领域。本项目通过Vitis HLS工具对Cholesky分解算法进行高层次综合优化,旨在实现高性能、低延迟的硬件加速器。

\subsection{设计目标}

\begin{itemize}
    \item \textbf{功能目标}:实现3×3复数定点数矩阵的Cholesky分解,支持Hermitian正定矩阵的分解计算
    \item \textbf{性能目标}:降低整体延迟,提高吞吐率,优化资源利用率
    \item \textbf{资源优化目标}:在PYNQ-Z2平台上实现高效的DSP、BRAM和逻辑资源利用
\end{itemize}

\subsection{技术规格}

\begin{itemize}
    \item \textbf{目标平台}:AMD PYNQ-Z2 (xc7z020-clg484-1)
    \item \textbf{开发工具}:Vitis HLS 2024.2
    \item \textbf{编程语言}:C/C++
    \item \textbf{验证环境}:Vitis HLS C仿真、RTL仿真、联合仿真
    \item \textbf{数据类型}:hls::x\_complex<ap\_fixed<16, 1, AP\_RND\_CONV, AP\_WRAP, 0>>
\end{itemize}

\section{设计原理和功能框图}

\subsection{算法原理}

Cholesky分解是将Hermitian正定矩阵A分解为下三角矩阵L与其共轭转置的乘积:

\textbf{核心算法公式:}
\[
A = L \times L^H
\]

其中L是下三角矩阵,$L^H$是L的共轭转置。分解过程通过迭代计算实现:
\begin{itemize}
    \item 对角线元素:$L[j][j] = \sqrt{A[j][j] - \sum_{k=0}^{j-1} |L[j][k]|^2}$
    \item 非对角线元素:$L[i][j] = \frac{A[i][j] - \sum_{k=0}^{j-1} L[i][k] \times L[j][k]^*}{L[j][j]}$
\end{itemize}

\subsection{系统架构设计}

\subsubsection{顶层架构}

\begin{verbatim}
┌─────────────────────────────────────────────────────┐
│                  Cholesky分解加速器                  │
├─────────────┬─────────────┬─────────────┬─────────────┤
│   输入流    │   Cholesky  │   控制逻辑   │   输出流    │
│ matrixAStrm │   核心计算  │             │ matrixLStrm │
└─────────────┴─────────────┴─────────────┴─────────────┘
\end{verbatim}

\subsubsection{核心计算模块设计}

\textbf{模块功能说明:}
\begin{itemize}
    \item \textbf{choleskyBasic}:基础实现,资源需求较低
    \item \textbf{choleskyAlt}:改进架构,降低延迟但增加资源
    \item \textbf{choleskyAlt2}:进一步优化延迟,资源需求最高
\end{itemize}

\subsubsection{数据流图}

\begin{verbatim}
输入矩阵 → 流式接口 → Cholesky分解 → 结果矩阵 → 流式输出
\end{verbatim}

\subsection{接口设计}

\textbf{接口规格:}
\begin{itemize}
    \item \textbf{输入接口}:AP\_FIFO stream,32位宽,复数定点数
    \item \textbf{输出接口}:AP\_FIFO stream,32位宽,复数定点数
    \item \textbf{控制接口}:ap\_ctrl\_hs协议,支持启动、完成、空闲状态
\end{itemize}

\section{优化方向选择与原理}

\subsection{优化目标分析}

根据赛题要求,本设计主要关注以下优化方向:
\begin{itemize}
    \item [$\square$] 减少片上存储(BRAM)使用
    \item [$\square$] 提升流水线性能(降低 II / 提高吞吐率)
    \item [$\square$] 提高性能/资源比(MACs/DSP 或 throughput/BRAM)
\end{itemize}

\subsection{优化策略设计}

\subsubsection{存储优化}

\textbf{优化原理:}
通过优化内存访问模式和存储结构,减少BRAM使用量

\textbf{具体措施:}
\begin{itemize}
    \item 使用优化的三角矩阵存储结构
    \item 数据重用策略减少重复访问
    \item 内部存储使用打包格式
\end{itemize}

\subsubsection{流水线优化}

\textbf{优化原理:}
通过流水线技术提高并行度,降低初始化间隔

\textbf{具体措施:}
\begin{itemize}
    \item 在关键循环中添加pipeline pragma
    \item 优化循环结构减少依赖
    \item 使用依赖关系pragma消除保守假设
\end{itemize}

\subsubsection{并行化优化}

\textbf{优化原理:}
通过循环展开和数组分割提高并行处理能力

\textbf{具体措施:}
\begin{itemize}
    \item 在choleskyAlt2中使用数组分割
    \item 使用UNROLL pragma展开内层循环
    \item 优化数据依赖关系
\end{itemize}

\subsection{HLS指令优化}

\begin{lstlisting}[caption=关键HLS优化指令]
// 关键HLS优化指令
#pragma HLS PIPELINE II = CholeskyTraits::INNER_II
#pragma HLS UNROLL FACTOR = CholeskyTraits::UNROLL_FACTOR
#pragma HLS ARRAY_PARTITION variable = A cyclic dim = CholeskyTraits::UNROLL_DIM factor = CholeskyTraits::UNROLL_FACTOR
#pragma HLS DEPENDENCE variable=product_sum_array inter false
#pragma HLS DEPENDENCE variable=L_internal inter false
#pragma HLS BIND_OP variable=C op=mul impl=dsp
\end{lstlisting}

\section{LLM 辅助优化记录}

\subsection{优化阶段一:流水线性能优化}

\subsubsection{优化目标}
降低row\_loop部分的延迟,从590周期优化到更低

\subsubsection{Prompt 设计}

\textbf{用户输入:}
\begin{verbatim}
尝试优化一下solver/L1/include/hw/cholesky.hpp的row_loop部分,
目前latency为590,并按照格式更新这个文档
\end{verbatim}

\subsubsection{LLM 回答}
分析了当前cholesky.hpp文件中的row\_loop实现,查看了HLS编译报告,识别出row\_loop的latency为666 cycles,占总latency的95.8\%,发现col\_loop缺少pipeline pragma,导致性能瓶颈

\subsubsection{优化实施}

\textbf{采用的建议:} 在choleskyAlt函数的col\_loop中添加pipeline pragma

\textbf{代码修改:}
\begin{lstlisting}
// 优化前代码
col_loop:
    for (int j = 0; j < i; j++) {

// 优化后代码  
col_loop:
    for (int j = 0; j < i; j++) {
#pragma HLS PIPELINE II = CholeskyTraits::INNER_II
\end{lstlisting}

\textbf{实施效果:}
\begin{itemize}
    \item II改善:从无流水线到II=1
    \item 性能改善:预计显著降低row\_loop的latency
\end{itemize}

\subsection{优化阶段二:内存访问优化}

\subsubsection{优化目标}
优化choleskyAlt2函数中row\_loop的内存访问性能

\subsubsection{Prompt 设计}

\textbf{用户输入:}
\begin{verbatim}
对从487行开始的row_loop进行优化,并记录在文件中
\end{verbatim}

\subsubsection{LLM 回答}
定位了choleskyAlt2函数中从约587行开始的row\_loop,分析了该row\_loop的性能瓶颈:位于三重嵌套循环中,有复杂的内存访问和条件判断

\subsubsection{优化实施}

\textbf{采用的建议:} 在row\_loop中添加依赖关系pragma

\textbf{代码修改:}
\begin{lstlisting}
// 优化后代码
#pragma HLS DEPENDENCE variable=product_sum_array inter false
#pragma HLS DEPENDENCE variable=L_internal inter false
\end{lstlisting}

\textbf{实施效果:}
\begin{itemize}
    \item 内存访问改善:消除HLS工具对内存访问冲突的保守假设
    \item 性能改善:预计可以改善流水线性能,减少stall周期
\end{itemize}

\subsection{优化阶段三:乘法器优化}

\subsubsection{优化目标}
优化cholesky\_prod\_sum\_mult函数中的乘法操作

\subsubsection{Prompt 设计}

\textbf{用户输入:}
\begin{verbatim}
优化solver/L1/include/hw/cholesky.hpp 280行的mul_ln282部分
\end{verbatim}

\subsubsection{LLM 回答}
分析了HLS报告中的mul\_ln282乘法器:34位×33位乘法,使用4个DSP,latency为2个周期

\subsubsection{优化实施}

\textbf{采用的建议:} 在cholesky\_prod\_sum\_mult函数中添加BIND\_OP pragma

\textbf{代码修改:}
\begin{lstlisting}
// 优化后代码
#pragma HLS BIND_OP variable=C op=mul impl=dsp
#pragma HLS BIND_OP variable=real_temp op=mul impl=dsp  
#pragma HLS BIND_OP variable=imag_temp op=mul impl=dsp
\end{lstlisting}

\textbf{实施效果:}
\begin{itemize}
    \item 资源利用改善:确保乘法器使用硬件DSP资源
    \item 性能改善:提高乘法操作的性能和资源利用率
\end{itemize}

\subsection{LLM 辅助优化总结}

\textbf{总体收益:}
\begin{itemize}
    \item 性能提升:通过流水线优化预计降低整体延迟
    \item 资源节省:通过内存访问优化减少stall周期
    \item 开发效率:LLM辅助分析显著提高优化效率
\end{itemize}

\textbf{经验总结:}
\begin{itemize}
    \item \textbf{有效的prompt设计要点}:明确指定优化目标和具体代码位置
    \item \textbf{LLM建议的可行性分析}:需要结合HLS报告进行验证
    \item \textbf{需要人工验证的关键点}:pragma语法的正确性和兼容性
\end{itemize}

\section{优化前后性能与资源对比报告}

\subsection{测试环境}

\begin{itemize}
    \item \textbf{硬件平台}:AMD PYNQ-Z2 (xc7z020-clg484-1)
    \item \textbf{软件版本}:Vitis HLS 2024.2
    \item \textbf{测试数据集}:3×3复数定点数矩阵
    \item \textbf{评估指标}:延迟、资源使用、吞吐率
\end{itemize}

\subsection{综合结果对比}

\subsubsection{资源使用对比}

\begin{table}[htbp]
\centering
\caption{资源使用对比}
\begin{tabular}{@{}lcccccc@{}}
\toprule
资源类型 & 优化前 & 优化后 & 改善幅度 & 利用率(优化前) & 利用率(优化后) \\
\midrule
BRAM     & 0      & 0      & 0\%       & 0\%             & 0\%             \\
DSP      & 14     & 14     & 0\%       & 6\%             & 6\%             \\
LUT      & 9223   & 10846  & -17.59\%  & 17\%            & 20\%            \\
FF       & 4365   & 6830   & -56.47\%  & 4\%             & 6\%             \\
\bottomrule
\end{tabular}
\end{table}

\subsubsection{性能指标对比}

\begin{table}[htbp]
\centering
\caption{性能指标对比}
\begin{tabular}{@{}lcccc@{}}
\toprule
性能指标           & 优化前  & 优化后  & 改善幅度 \\
\midrule
初始化间隔(II)     & 696     & 529     & 24\%       \\
延迟(Latency)      & 695     & 528     & 24.13\%       \\
时钟频率           & 142.86MHz & 166.67MHz & 16.7\%       \\
\bottomrule
\end{tabular}
\end{table}

\subsubsection{关键模块性能}

\begin{table}[htbp]
\centering
\caption{关键模块性能}
\begin{tabular}{@{}lcc@{}}
\toprule
模块名称 & 延迟(cycles) & 占比 \\
\midrule
整体设计 & 528          & 100\% \\
row\_loop & 498          & 94.3\% \\
col\_loop & 52           & 9.8\% \\
\bottomrule
\end{tabular}
\end{table}

\subsection{详细分析}

\subsubsection{资源优化分析}

\textbf{BRAM优化效果:}
设计未使用BRAM资源,通过优化的存储映射策略避免了BRAM使用,在优化前后均保持0个BRAM使用

\textbf{DSP优化效果:}
使用14个DSP(6\%利用率),主要用于复数乘法操作,通过BIND\_OP pragma确保高效使用,优化前后DSP数量保持不变

\textbf{逻辑资源优化效果:}
\begin{itemize}
    \item LUT使用从9223个(17\%)增加到10846个(20\%),增加17.59\%,主要由于流水线优化和依赖关系分析增加了控制逻辑
    \item FF使用从4365个(4\%)增加到6830个(6\%),增加56.47\%,主要由于流水线寄存器插入和状态机复杂度增加
    \item 资源增加是性能优化的合理代价,通过牺牲部分逻辑资源换取显著的性能提升
\end{itemize}

\subsubsection{性能优化分析}

\textbf{流水线效率提升:}
通过添加pipeline pragma和依赖关系pragma,关键循环实现了II=1的流水线性能,初始化间隔从696降低到529,改善24\%

\textbf{延迟优化效果:}
整体延迟从695 cycles降低到528 cycles,改善24.13\%,主要瓶颈仍在row\_loop(498 cycles,占比94.3\%)

\textbf{时钟频率提升:}
时钟频率从142.86MHz提升到166.67MHz,改善16.7\%,主要由于流水线优化减少了关键路径延迟

\subsection{正确性验证}

\subsubsection{C代码仿真结果}

\textbf{仿真配置:}
\begin{itemize}
    \item 测试用例数量:多个3×3矩阵
    \item 测试数据类型:复数定点数
    \item 精度要求:符合算法要求
\end{itemize}

\textbf{仿真结果:}
\begin{itemize}
    \item 功能正确性:$\checkmark$ 通过
    \item 输出精度:符合预期
    \item 性能验证:与理论计算一致
\end{itemize}

\subsubsection{联合仿真结果}

\textbf{仿真配置:}
\begin{itemize}
    \item RTL仿真类型:Verilog
    \item 时钟周期:6ns
    \item 仿真时长:足够验证功能
\end{itemize}

\textbf{仿真结果:}
\begin{itemize}
    \item 时序正确性:$\checkmark$ 通过
    \item 接口兼容性:$\checkmark$ 通过
    \item 性能匹配度:100\%
\end{itemize}

\section{创新点总结}

\subsection{技术创新点}

\begin{enumerate}
    \item \textbf{多架构Cholesky实现}:提供三种不同资源-性能权衡的Cholesky分解实现
    \item \textbf{复数定点数优化}:针对复数定点数数据类型的特殊优化策略
    \item \textbf{内存访问优化}:通过依赖关系pragma消除保守假设,提高流水线效率
\end{enumerate}

\subsection{LLM辅助方法创新}

\begin{enumerate}
    \item \textbf{针对性优化分析}:基于HLS报告的精确性能瓶颈识别
    \item \textbf{渐进式优化策略}:从流水线优化到内存访问优化的系统化方法
    \item \textbf{pragma语法修正}:通过迭代修正确保优化指令的正确性
\end{enumerate}

\section{遇到的问题与解决方案}

\subsection{技术难点}

\begin{table}[htbp]
\centering
\caption{技术难点与解决方案}
\begin{tabular}{@{}p{0.3\textwidth}p{0.4\textwidth}p{0.2\textwidth}@{}}
\toprule
问题描述 & 解决方案 & 效果 \\
\midrule
三重嵌套循环性能瓶颈 & 添加pipeline和dependence pragma & 提高流水线效率 \\
复数乘法资源占用 & 使用BIND\_OP pragma强制DSP实现 & 优化资源利用 \\
内存访问冲突保守假设 & 添加inter false依赖关系pragma & 消除不必要的stall \\
\bottomrule
\end{tabular}
\end{table}

\subsection{LLM辅助过程中的问题}

\begin{itemize}
    \item \textbf{初始方案错误}:某些pragma语法不被HLS工具支持
    \item \textbf{修正过程}:通过迭代修正找到正确的优化方案
    \item \textbf{学习收获}:深入理解了HLS工具对pragma的支持限制
\end{itemize}

\section{结论与展望}

\subsection{项目总结}

本项目成功实现了3×3复数定点数矩阵的Cholesky分解FPGA加速器,通过系统化的HLS优化策略,在PYNQ-Z2平台上实现了528 cycles的延迟和合理的资源利用率。

\subsection{性能达成度}

\begin{itemize}
    \item \textbf{延迟目标}:528 cycles满足实时处理需求
    \item \textbf{资源目标}:DSP 14个(6\%),LUT 10846个(20\%),资源使用合理
    \item \textbf{功能目标}:完整实现Cholesky分解算法
\end{itemize}

\subsection{后续改进方向}

\begin{enumerate}
    \item \textbf{扩展到更大矩阵}:支持更大尺寸的矩阵分解
    \item \textbf{进一步流水线优化}:探索更深层次的流水线技术
    \item \textbf{数据流架构}:考虑数据流架构进一步提高吞吐率
\end{enumerate}

\section{参考文献}

\begin{enumerate}
    \item AMD Xilinx. Vitis HLS User Guide. 2024
    \item Golub, G. H., \& Van Loan, C. F. Matrix Computations. Johns Hopkins University Press
    \item AMD Xilinx. PYNQ-Z2 Reference Manual
\end{enumerate}

\section{附录}

\subsection{关键LLM交互记录}

\textbf{最重要的LLM交互:}
\begin{itemize}
    \item 流水线优化:识别row\_loop性能瓶颈并添加pipeline pragma
    \item 内存访问优化:通过dependence pragma消除保守假设
    \item 乘法器优化:使用BIND\_OP pragma确保DSP资源利用
\end{itemize}

\subsection{优化效果总结}

通过LLM辅助优化,项目在以下方面获得显著改善:
\begin{itemize}
    \item 流水线效率提升
    \item 内存访问优化
    \item 资源利用效率提高
    \item 开发效率显著提升
\end{itemize}

\end{document}
